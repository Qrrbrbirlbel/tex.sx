\documentclass{beamer}
\usepackage{tikz}
\usetikzlibrary{intersections,through}
\makeatletter
\newenvironment{xcenter}
 {\par\setbox0=\hbox\bgroup\ignorespaces}
 {\unskip\egroup\noindent\makebox[\textwidth]{\box0}\par}
\tikzset{
 invisible/.style={opacity=0},
 id/.style={visible on=<#1->},
 visible on/.style={alt={#1{}{invisible}}},
 alt/.code args={<#1>#2#3}{%
   \alt<#1>{\pgfkeysalso{#2}}{\pgfkeysalso{#3}}}}
\setbeamertemplate{navigation symbols}{}
\def\info#1#2{\only<#1>{#1: #2}\ignorespaces}
\tikzset{% the coordinate ([scale around=f:(a)] b) is the same as ($(a)!f!(b)$)
  xscale around/.code=%
    \tikz@addtransform{\def\tikz@aroundaction{\pgftransformxscale}\tikz@doaround{#1}},
  yscale around/.code=%
    \tikz@addtransform{\def\tikz@aroundaction{\pgftransformyscale}\tikz@doaround{#1}},
  scale/.style={xscale={#1}, yscale={#1}}}
\makeatother
\begin{document}
\begin{frame}\begin{xcenter}
\begin{tikzpicture}[ni/.style={name intersections={####1}}, intersection/name=i, x=+.7cm, y=+.7cm,
  thick, line cap=round, line join=round, declare function={a=7;b=8;c=6;}, remember picture,
  circle through/.append style={overlay, help lines, draw}, label position=below,
  nl/.style={shape=coordinate, name={####1}, label={$####1$}}]
\def\helppath{\draw[overlay,help lines]}
\scope
\path[id=2] node[nl=A] {} node [visible on=<3-6>, circle through=++(right:b), name path=cAC] {}
  ++ (right:c) node[nl=B] {} node [visible on=<3-6>, circle through=++(right:a), name path=cBC] {}
                         (A) node [visible on=<5-6>, circle through=(B), name path=cAB] {};
\draw[ni={of={cAC and cBC}, by=C}]
  (A) edge[id=2] (B) (B) edge[id=4] (C) (C) edge[id=4] node[at start, above] {$C$} (A);
\helppath (A) -- ([scale around=2:(A)] B) [name path=lAB];\path([scale around=2:(A)] B);
\helppath[visible on=<4-6>] (A) -- ([scale around=2:(A)] C) [name path=lAC];

\path[id=6, every label/.append style={id=6},label position=above left][ni={of=lAC and cAB, by={[nl=B']}}] (i-1) coordinate (B')
     [label position=below]     [ni={of=lAB and cAC, by={[nl=C']}}] (i-1) coordinate (C');
\helppath[id=6] ([scale around=2:(B')] C') -- ([scale around=2:(C')] B') [name path=lC'B'];
\helppath[id=7] (C) node [circle through={([scale around=2:(C)] B')}, name path=cC] {};

%\tikzset{every path/.append style={invisible}}
\helppath[id=8,ni={of=cC and lC'B', name=CB'}]
  (CB'-1) node[circle through=(C), name path=cCB'-1] {}
  (CB'-2) node[circle through=(C), name path=cCB'-2] {};
\draw[visible on=<8>,overlay] (CB'-1) circle (1pt) (CB'-2) circle (1pt);
\helppath[id=9, ni={of={cCB'-1 and cCB'-2}}] [name path=lOrthoC]
     ([scale around=3:(i-2)] i-1) -- ([scale around=2:(i-1)] i-2);
\draw[visible on=<9>,overlay] (i-1) circle (1pt) (i-2) circle (1pt);
\helppath[id=11,overlay,ni={of=cC and lOrthoC}] (i-1) node[circle through=(i-2), name path=cC'-1] {}
                              (i-2) node[circle through=(i-1), name path=cC'-2] {};
\draw[visible on=<10-11>] (i-1) circle (1pt) (i-2) circle (1pt);
\helppath[id=12,  ni={of=cC'-1 and cC'-2}] ([scale around=1.2:(C)]i-1) -- ([scale around=1.3:(C)] intersection of A--B and i-1--i-2);
\draw[id=13, ni={of=cC'-1 and cC'-2}] % it's happening!
  (C) -- (intersection of A--B and i-1--i-2) node[nl=C'']{} -- (B);

\draw[visible on=<12>,overlay, latex-] (current page.north-|i-1) -- ++(down:1cm) node[right]{The other one is up here!};
\draw[visible on=<12>,overlay] (i-2) circle (1pt);
\endscope
\node[overlay,align=center,anchor=north,font=\strut,text depth=+0pt,fill=white] at (6,-1) {
  \info{1}{Let's start with one side somewhere}
  \info{2}{Draw first side}
  \info{3}{Draw circles with length of other sides around $A$ and $B$}
  \info{4}{Intersection is $C$}
  \info{5}{We need that circle around $A$ that goes through $B$, too}
  \info{6}{Our goal: A line that is parallel to $\overline{B'C'}$ \emph{and} goes through $C$}
  \info{7}{Draw a circle around $C$ that intersects $\overline{B'C'}$ at least twice}
  \info{8}{Draw circles where these circle and $\overline{B'C'}$ intersect\\
  through $C$ so that they have the same radius}
  \info{9}{Draw a line between the points where these circles intersect}
  \info{10}{This line intersects our first circle (around $C$) twice}
  \info{11}{Draw circles around these points through the other points respectively\\
    so that they have the same radius}
  \info{12}{Draw a line between the points where these circles intersect and\\
    you have a parallel line to $\overline{B'C'}$ through $C$}
  \info{13}{Where this line intersects the line $\overline{AB}$, you have $C''$}
};
\end{tikzpicture}\end{xcenter}\end{frame}
\end{document}
